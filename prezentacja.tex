\documentclass[3mm]{beamer}

\usepackage[utf8]{inputenc} 			%Pakiet kodowania znaków
\usepackage[OT4]{polski} 				%Pakiet języka polskiego
\usepackage[polish]{babel}

\usepackage{graphicx}
\usepackage{tikz}

\definecolor{mdgray}{RGB}{100,105,100}

\title{\centering\color[RGB]{1,5,1}\emph{\fontfamily{qcs}\selectfont
\hfill\\\hfill\\\hfill\\\hfill\\Transmisja w systemie FEC \\ Forward Error Correction \\
\noindent\rule{4cm}{0.4pt}}}

\author{\color{mdgray}\fontfamily{qtm}\selectfont Weronika Mrugała, Adam Szcześniak,\\Adam Cierniak}



\date{}

\usefonttheme{serif}
\setbeamercolor{frametitle}{fg=black}
\setbeamercolor{structure}{fg=black}
\setbeamercolor{normal text}{fg=black}
%\setbeamerfont{frametitle}{\emph}
\setbeamercolor{frametitle}{bg=mdgray}
\setbeamercolor{frametitle}{fg=white}


\begin{document}

\begin{frame}
	\tikz [remember picture,overlay]
    \node at
        ([yshift=3cm]current page.center) 
        %or: (current page.center)
        {\includegraphics[scale=0.1]{logotyp/logo_PWr_czarne_poziom__bez_tla.png}};
	\maketitle
\end{frame}

\begin{frame}
	\frametitle{{Zawartość prezentacji}}
	\setcounter{tocdepth}{2}
	\pagenumbering{arabic}
	\tableofcontents
\end{frame}

\section{Wstęp}	
\begin{frame}{Wstęp}
	
	Czym są kody korekcyjne, kiedy się je stosuje, historia.
\end{frame}

\section{Przykłady kodów FEC}
	\subsection{Potrajanie bitów}
\begin{frame}
	\subsection{BCH}
	\subsection{Kodowanie Reeda-Salomona}
	\section{Porównanie}
\end{frame}
\note{Nie mam pojęcia jak to numerować}

\end{document}