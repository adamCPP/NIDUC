\documentclass[12pt]{beamer}

\usepackage[utf8]{inputenc} 			%Pakiet kodowania znaków
\usepackage[OT4]{polski} 				%Pakiet języka polskiego
\usepackage[polish]{babel}

\usepackage{graphicx}
\usepackage{tikz}

\definecolor{mdgray}{RGB}{100,105,100}

\title{\centering\color[RGB]{1,5,1}\emph{\fontfamily{qcs}\selectfont
\hfill\\\hfill\\\hfill\\\hfill\\Transmisja w systemie FEC \\ Forward Error Correction \\
\noindent\rule{4cm}{0.4pt}}}

\author{\color{mdgray}\fontfamily{qtm}\selectfont Weronika Mrugała, Adam Szcześniak,\\Adam Cierniak}

\date{}

\usefonttheme{serif}
\setbeamertemplate{section in toc}[sections numbered]
\setbeamertemplate{itemize item}[circle]

%Kiepsko działa to poniższe
%\setbeamercolor{frametitle}{fg=black}
\setbeamercolor{structure}{fg=black}
\setbeamercolor{normal text}{fg=black}
\setbeamercolor{frame}{bg=mdgray}
%\setbeamerfont{frametitle}{\emph}
\setbeamercolor{frametitle}{bg=mdgray}
%\setbeamercolor{frametitle}{fg=white}

%Białe 'j' aby utrzymać linię \hrule na stałym poziomie
\setbeamertemplate{frametitle}{ 
        \color{mdgray}{\hspace{2ex}\emph{\insertframetitle}\color{white}{j}\color{mdgray} \hfill {\ifnum0= \insertsectionnumber {} \else \emph{\large\insertsectionnumber} \fi} \hspace{1ex} \hrule}}


\begin{document}

\begin{frame}
	\tikz [remember picture,overlay]
    \node at
        ([yshift=-1.5cm]current page.north) 
        {\includegraphics[scale=0.12]{logotyp/logo_PWr_czarne_poziom__bez_tla.png}};
	\maketitle
\end{frame}

\pagenumbering{arabic}

\begin{frame}{Plan prezentacji}
	\setcounter{section}{0}
	\tableofcontents
\end{frame}

\section{Wstęp}	
\setcounter{section}{1}
\begin{frame}{Wstęp}	
	Czym są kody korekcyjne, kiedy się je stosuje, historia, wprowadzenie pojęć.
\end{frame}

\section{Przykłady kodów FEC}
	\subsection{Potrajanie bitów}
\begin{frame}{Potrajanie bitów}
	
\end{frame}
\note{Tak można robić notatki, które później są widoczne na komputerze, ale nie rzutniku. Trzeba tylko ogarnąć jak to się dokładnie robi.}

\subsection{BCH}
\begin{frame}{BCH}

\end{frame}
	
\subsection{Kodowanie Reeda-Salomona}
\begin{frame}{Kodowanie Reeda-Salomona}

\end{frame}

\section{Porównanie}
\begin{frame}{Porównanie}
Jakaś tabela porównawcza kodów korekcyjnych:
\begin{itemize}
		\item blokowe, czy jest przeplot
		\item sprawność (współczynnik) kodowania k/n
		\item możliwości detekcyjne i korekcyjne, dmin?
		\item systematyczne czy niesystematyczne?
		\item binarny/niebinarny???
		\item zysk kodowy w dB???
		\item odporność na błędy seryjne
		\item \ldots
\end{itemize}
		
\end{frame}

\section{Bibliografia}
\begin{frame}{Bibliografia}
	
\end{frame}

\section*{Pytania}
\begin{frame}
	\centering \Large\emph{Dziękujemy za uwagę. Pytania?}
\end{frame}

\end{document}